\documentclass{a0poster}
\usepackage{fancytikzposter} 

\setmargin{2.25}
\setblockspacing{1.4}
\setblocktitleheight{1.5}
\setcolumnnumber{3}
\usetemplate{3}
\setheaddrawingheight{14}


\usepackage[margin=\margin cm, paperwidth=118.9cm, paperheight=84.1cm]{geometry} 

\usepackage{cmbright}
\usepackage[math]{kurier}
\usepackage[T1]{fontenc}

\usepackage{pgfplots}
\usepackage{tabularx}
\newcommand{\bi}{\begin{itemize}}
\newcommand{\ei}{\end{itemize}}
\newcommand{\dst}{\displaystyle}

\def\imagetop#1{\vtop{\null\hbox{#1}}}

\title{\huge Effects of locality and mixing in threshold models of social behavior}
\author{\underline{Andrei R. Akhmetzhanov}$\,{}^{1,2}$, Lee Worden$\,{}^2$, Jonathan Dushoff$\,{}^{2,*}$\\
  \quad ${}^1$ Inst. of Evolutionary Sciences, University of Montpellier 2, France;\ \ ${}^2$ Dept. of Biology, McMaster University, Canada\\
  ${}^*$ Corresponding author, \texttt{dushoff@mcmaster.ca}
}

\begin{document}

%%%%% ---------- the background picture ---------- %%%%%
\ClearShipoutPicture
\AddToShipoutPicture{\BackgroundPicture}

\noindent % to have the picture right in the center
\begin{tikzpicture}
  \initializesizeandshifts
  
  \titleblock[(0,.3)]{60}{1.5}
  \addlogo[south west]{(-10,4.25)}{10cm}{um2_logo.eps}
  \addlogo[south west]{(.812\linewidth,1.9)}{7cm}{mac_logo.eps}

  \blocknode{Motivation} {

    % no space!
    %\begin{center}
    %  \begin{tabular}[t]{cc}
    %    \begin{minipage}{0.3\linewidth}
    %      \includegraphics[width=.9\textwidth]{japanese_women.jpg}
    %    \end{minipage}
    %    & 
    %    \begin{minipage}{0.6\linewidth}
    %      The Japanese hide their mouths when they laugh, because it's indecent to show the laughter \\
    %      Grand magasin, Tokyo, 1958
    %    \end{minipage}
    %  \end{tabular}
    %\end{center}

    \coloredbox{colorthree!50!}{Footbinding tradition in China}
    
    \begin{center}
      \begin{tabular}[t]{cc}
        \begin{minipage}{0.25\linewidth}
          \includegraphics[width=.9\textwidth]{child-feet-bound.jpg}
        \end{minipage}
        & 
        {\small \begin{minipage}{0.65\linewidth}
          \bi 
            \item Appeared in the Sung Dynasty (960-1279) and was widespread for thousands of years
            \item Caused serious pain and health risks to women (complications included ulceration, paralysis, gangrene)
            \item Treated women were crippled and largely housebound
            \item May have begun as patriarchal practice to ``ensure paternity confidence'' and persisted as a simple coordination problem
            \item The tradition was defended even by women and was transmitted by them in large cities
          \ei\medskip

          \bi
            \item Abandoned in a single generation. Ended between 1900 and 1911
            \item The main unit of spread for the abandonment movement was large cities. It was initiated in Shanghai
            \item Programme content was: health disadvantages, outside world not practicing, advantages of natural feet
          \ei
        \end{minipage}}
      \end{tabular}
    \end{center}
    
    \coloredbox{colorthree!50!}{Female genital cutting}

    \begin{center}
      \begin{tabular}[t]{cc}
        \begin{minipage}{0.25\linewidth}
          \includegraphics[width=.9\textwidth]{girl_after_FGC.jpg}\\
          \includegraphics[width=.9\textwidth]{cutter.png}

          \centerline{\tiny \color{blue}[(c) Stephanie Sinclair \& Reuters]}
        \end{minipage}
        & 
        {\small\begin{minipage}{0.65\linewidth} 
          \bi
            \item Is widespread in Africa and some Asian countries, being prevalent in some countries -- including Egypt (91\%) and Sudan (89\%) {\color{blue} [UNICEF 2008,2010]}
            \item 100-130 millions of women, currently living who have undergone FGC, many at a very young age {\color{blue} [WHO 2012]}
          \ei\medskip

          \bi
            \item Causes serious pain and health risks to women
            \item Has been widespread for thousands of years
            \item May have begun as patriarchal practice to ``ensure paternity confidence'' and persisted as a simple coordination problem
            \item Can it also go away quickly?
          \ei
        \end{minipage}}
      \end{tabular}
    \end{center}

    \coloredbox{colorthree!50!}{Social norm as coordination problem}

    \centerline{\begin{minipage}{0.95\linewidth}
      \innerblock{Social norm}{
        \bi 
          \item Individuals wouldn't do it alone 
          \item They'll do it if enough other people do it 
        \ei}
    \end{minipage}}
  }

  \blocknode{Threshold model for coordination problems} {
  
    \coloredbox{colorthree!50!}{Model {\color{blue} [Schelling 1971; Granovetter 1978]}}

    \centerline{\begin{minipage}{0.95\linewidth}
        \bi 
          \item Personal beliefs, interests, access to resources, social roles, circumstances, {\it etc.}\ all influence people's tendency to \textbf{adopt} or \textbf{reject} a convention.
          \item Probably the simplest model for social norm: idealize them all in a \textbf{threshold}.
          \item adopt if current level of adoption exceeds the threshold; reject if not 
        \ei
    \end{minipage}}\medskip

    \coloredbox{colorthree!50!}{Contact networks}
    
    \centerline{\begin{minipage}{0.95\linewidth}
        \bi 
          \item People aren't influenced equally by everyone everywhere: they have a limited number of contacts
          \item Consider spread of adoption/rejection on lattices and networks
          \item With and without mixing
        \ei
    \end{minipage}}
  }

  %%%%%%%%%%%%% NEW COLUMN %%%%%%%%%%%%%%% 
  \startsecondcolumn 

  %%%%%%%%%% ------------------------------------------ %%%%%%%%%%
  \blocknode{Fast mixing: analytics} {

  \coloredbox{colorthree!50!}{Infinite neighborhood size} 

  \begin{tabular}[h]{cc}
    \begin{minipage}{0.65\linewidth}
      \centerline{\includegraphics[width=.9\linewidth]{granovetter1.pdf}}
    \end{minipage}
    &
    \raisebox{.5cm}{\begin{minipage}{0.3\linewidth}
      {\small Gaussian PDF $f(x)$ \\ mean $0.65$ and s.d. {\color{blue} \bf 0.15}, {\color{brown} \bf 0.25}\medskip

      Curve on the right: \\ CDF $F(y) = \int^y_{-\infty} f(\xi)\>\mathrm d\xi$\medskip

      Equilibrium points {\color{blue} $y_-<y_*<y_+$} = {\color{green} \bf Green}}
    \end{minipage}}
  \end{tabular}\bigskip

  %\coloredbox{colorthree!50!}{Discrete {\it vs.} continuous time}

  \begin{tabular}[h]{cc}
    \hspace*{-.7cm}\begin{minipage}{0.65\linewidth}
      \centerline{\includegraphics[width=.9\linewidth]{granovetter2.pdf}}
    \end{minipage}
    &
    \begin{minipage}{0.3\linewidth}
      {\small {\bf Discrete time}: synchronous updates, {\color{blue} $y(t+1)=F(y(t))$}\medskip

      {\bf Cont. time}: asynchronous updates, {\color{blue} $\mathrm dy/\mathrm dt = F(y)-y$}\medskip}
    \end{minipage}
  \end{tabular}\bigskip

  \coloredbox{colorthree!50!}{Finite neighborhood size}

  \begin{center}
    \begin{tabular}[h]{lc}
      \begin{minipage}{.3\linewidth}
        \includegraphics[width=.9\linewidth]{finite_nhood_fast_mixing.pdf}
      \end{minipage}
      &
      \begin{minipage}{.6\linewidth}
        {\small {\bf Activation curve}: {\color{blue} $F_\mathcal P = \sum_{n=0}^\infty\mathcal P(n)F_n(y)$}, where {\color{blue} $P(n)$} is the probability distribution of connections in the network {\color{blue} $\mathcal P$} and \\
        \centerline{\color{red} $F_n(y) = \sum\limits_{k=0}^nF\! \left(\frac kn \right)C_n^k y^k(1-y)^{n-k}\,,$}\\
        where {\color{blue} $C_n^k$} is a binomial coefficient\\[.25cm]
        {\bf Potential function}: {\color{blue} $V_n(y) = \int^y(F_n(\xi)-\xi)\>\mathrm d\xi$}\\[.3cm]
        {\bf In Fig}: nhoods = {\bf \color{magenta} 4}, {\bf \color{red} 12}, {\bf \color{green} 24}, {\bf \color{blue} $\infty$}}
      \end{minipage}
    \end{tabular}
  \end{center}
  }

  \blocknode{Zero mixing: first simulations} {
    \begin{tabular}[h]{cc}
      \begin{minipage}{0.4\linewidth}
        \includegraphics[width=.8\linewidth]{tsp-dynamics.pdf}
      \end{minipage}
      &    
      \hspace*{-1.5cm}{\raisebox{1.5cm}{\begin{minipage}{0.57\linewidth}
        2d-lattice of size $100\times100$.\smallskip

        Thresholds are normally distributed with the mean {\color{blue} 0.45} and standard deviation {\color{blue} 0.3}. The initial pattern of thresholds and initial states is the same for all simulations shown.\smallskip

        Initial activation level: \color{blue}{15\%}
      \end{minipage}}}
    \end{tabular}
  }

  \blocknode{Different mixing rates} {
    \begin{center}
      \begin{tabularx}{\linewidth}{>{\centering}X >{\centering}X >{\centering\arraybackslash}X}
        {\bf Zero mixing} {\color{blue} $\mu=0$} & {\bf Intermediate mixing} {\color{blue} $\mu\sim1$} & {\bf Fast mixing} {\color{blue} $\mu\gg1$} \\
        \includegraphics[width=.95\linewidth]{zero-mixing.pdf} & \includegraphics[width=.95\linewidth]{interm-mixing.pdf} & \includegraphics[width=.95\linewidth]{fast-mixing.pdf}\\
        Metastable equilibria & Slow-manifold (SM) & Activation curve = SM \\
        ({\small {\it e.g.} {\color{blue} [Rosinberg 2008]}}) & & ({\small Mean-field approximation})
      \end{tabularx}
    \end{center}
  }

  %%%%%%%%%%%%% NEW COLUMN %%%%%%%%%%%%%%% 
  %% (if column number is 3)
  \startthirdcolumn

  %%%%%%%%%% ------------------------------------------ %%%%%%%%%%
  \blocknode{Zero mixing: network structure \& metastability} {
    \begin{tabular}[h]{cc}
      \hspace*{2.25cm}{\begin{minipage}{0.32\linewidth}
        \begin{center}
          \includegraphics[width=.95\linewidth]{zero-mixing-networks.pdf}
        \end{center}
      \end{minipage}}
      &    
      \hspace{2.0cm}{\raisebox{1.75cm}{\begin{minipage}{0.65\linewidth}
        \underline{Same} size: {\color{blue} $800\times800$}, \underline{same} neighborhood: {\bf 8}\\[.4cm]

        {\bf Threshold distribution:} Gaussian with \\ the mean {\color{blue} 0.6} and st.dev. {\color{blue} 0.219}
      \end{minipage}}}
    \end{tabular}
  }

  \blocknode{Intermediate mixing: detailed study} {

     \begin{tabular}[h]{cc}
      \begin{minipage}{0.47\linewidth}
        \begin{center}
          \includegraphics[width=.8\linewidth]{interm-mixing-detailed.pdf}

          {\color{blue} $y_*^\mu$} -- unstable equibilibrium for {\color{blue} $\mu$},\\
          {\color{blue} $y_\times^\mu$} -- its corresponding initial point 
        \end{center}
      \end{minipage}
      &    
      \begin{minipage}{0.5\linewidth}
        \includegraphics[width=.8\linewidth]{prob_to_stay_in_shallow_well.pdf}
      \end{minipage}
    \end{tabular}\\[.4cm]

    \centerline{{\bf Bifurcation} at {\color{red} $\bar\mu$} -- the SM becomes tangent}

    \coloredbox{colorthree!50!}{Transition times}

    \begin{tabular}[h]{cc}
      \begin{minipage}{0.47\linewidth}
        \hspace*{1.75cm}{\includegraphics[width=.8\linewidth]{exit-times.pdf}}
      \end{minipage}
      &    
      \hspace*{-1.25cm}{\raisebox{.5cm}{\begin{minipage}{0.5\linewidth}
        {\small 
        Initial activation: {\bf 100\%}\\[.2cm]
        Average time to cross {\color{blue} $y_*^\infty$}, which is unstable equilibrium in fast-mixing limit {\color{blue} $\mu\gg1$}}\\[.25cm]

        \centerline{\begin{tabular}[h]{lc}
          \raisebox{.5cm}{\begin{minipage}{.45\linewidth}
            Association with\\ potential
          \end{minipage}}
          &
          \begin{minipage}{.3\linewidth}
            \includegraphics[width=.9\linewidth]{sketch.png}
          \end{minipage}
        \end{tabular}}
      \end{minipage}}}
    \end{tabular}

  }

  \blocknode{Conclusions} {
    \bi 
      \item Outcome can depend sensitively on distribution of thresholds
      \item \textbf{Neighborhood size} can change the outcome, even with very fast mixing
      \item Relaxation to global minimum is helped by \textbf{lower mixing rate}
      \item With finite mixing rates, there is a \textbf{SM}, which suggests a simplifying analysis
    \ei
  }

  \blocknode{Acknowledgment} {
    J.S. McDonnell Foundation, SHARCNET:www.sharcnet.ca
  }
  
  
\end{tikzpicture}

\end{document}
